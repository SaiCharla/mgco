\par The communication protocol between Ground Checkout and TDACS is text based
over TCP/IP. There are 2 types of data packets namely:
\begin{enumerate}
	\item Data Frame
	\item Data Base Frame
\end{enumerate}
A sample of varius data frames are shown as follows:
\lstinputlisting{gc_data.txt}

If F is a Ground checkout frame, the varius elements of the ground checkout
frame are listed as below:

% Frame details (Afterf stripping STX and ETX):
% frame[0] = frame type (DATA, DB)
% frame[1:18] = time stamp
% frame[18:22] = no. of parms
% frame[22:27] = white spaces
% frame[27:] = data
% if DATA: len = 8 bytes
% if DB: len = 25 bytes

\begin{itemize}
	\item F[0] : ASCII Start of Text (STX) : '0x02'
	\item F[1] : Byte designating the type of frame : '0xcc' - Data Base Frame,
		'0xbb' - Data Frame
	\item F[2:19] : Time stamp of the frame :  DDMMYYYYhhmmss000 (the trailing
		zeros are for milliseconds if used)
	\item F[19:23] : Number of parameters in the frame
	\item F[23:28] : Mandatory White Spaces
	\item F[28:-1] : Data : 8 bytes for each parametes if it is a Data Frame
		and 25 bytes for each parameter if it is Data Base Frame.
	\item F[-1] : ASCII End of Text (ETX) : '0x03'
\end{itemize}
The frame in encoded using ascii character code set.


