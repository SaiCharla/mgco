The main source code spreads across the following python files namely:
\begin{enumerate}
	\item pipeline\_server.py
	\item client\_branch.py
	\item protocol.py
	\item fileilockio.py
	\item config.py
\end{enumerate}
The confuguration of the application is set using 2 configuration files namely:
\begin{enumerate}
	\item GC.conf
	\item pipeline.conf
\end{enumerate}

\section{Configuration Files}

\subsection{GC.conf}
This file will contain the necessary details of the Ground Checkout servers
including hostname, port, no. of parameters and the sampling time. An example
of GC.conf is shown as below:
\lstinputlisting{../GC.conf}

\subsection{pipeline.conf}
This file contains the details required for setting up the pipeline server
including hostname, port and the sampling time. An example of pipeline.conf is
shown below:
\lstinputlisting{../pipeline.conf}


\section{Python Modules}

\subsection{config.py}
This file has 2 functions namely $get\_config()$ and $pipeline\_config()$ which
parse GC.conf and pipeline.conf respectively an return the configuration
details in proper object type and format.
%\begin{itemize}
%	\item get\_config([GC.conf]) $->$ int n, (str, int)\_array addrs, int\_array nparms,
%		int\_array st\_list
%	\item pipeline\_config([pipeline.conf]) $->$ (str, int)\_array addrs, int\_array
%		sleeptime
%\end{itemize}


\subsection{filelockio.py}
This file has 2 functions namely $filelockread()$ and $filelockwrite()$ which read
and write in to files with a lock in place while reading or writing to avoid
simultaneous reading and writhing by different processes.


\subsection{protocol.py}
This file contains the functions and classes for creating and manipulating the
Ground Checkout data frames. The various contents of the file are listed below:


\subsection{pipeline\_server}
This file contains the PipelineServer class which creates a PipelineServer
object which connects to
